\documentclass[10pt, aspectratio=169]{beamer}
% \documentclass[11pt,handout]{beamer}
\usepackage[T1]{fontenc}
\usepackage[utf8]{inputenc}
\usepackage{textcomp}
\usepackage{float, afterpage, rotating, graphicx}
\usepackage{epstopdf}
\usepackage{longtable, booktabs, tabularx}
\usepackage{fancyvrb, moreverb, relsize}
\usepackage{eurosym, calc}
\usepackage{amsmath, amssymb, amsfonts, amsthm, bm}
\usepackage[
    natbib=true,
    bibencoding=inputenc,
    bibstyle=authoryear-ibid,
    citestyle=authoryear-comp,
    maxcitenames=3,
    maxbibnames=10,
    useprefix=false,
    sortcites=true,
    backend=biber
]{biblatex}

% for figures
\usepackage{subcaption}
\usepackage[english]{babel}

\usepackage{graphicx}
\graphicspath{{../bld/python/figures/}}

% for csv tables
\usepackage{siunitx,array,booktabs}
\usepackage{csvsimple}
\usepackage{adjustbox}

\AtBeginDocument{\toggletrue{blx@useprefix}}
\AtBeginBibliography{\togglefalse{blx@useprefix}}
\setlength{\bibitemsep}{1.5ex}
\addbibresource{refs.bib}

\hypersetup{colorlinks=true, linkcolor=black, anchorcolor=black, citecolor=black, filecolor=black, menucolor=black, runcolor=black, urlcolor=black}

\setbeamertemplate{footline}[frame number]
\setbeamertemplate{navigation symbols}{}
\setbeamertemplate{frametitle}{\centering\vspace{1ex}\insertframetitle\par}
\setbeamertemplate{caption}[numbered]

\begin{document}

\title{Crime Data Mapping and Spatial Regression}

\author[Mudabbira Mushtary]
{
{\bf Mudabbira Mushtary}\\
{\small University of Bonn}\\[1ex]
}


\begin{frame}
    \titlepage
    \note{~}
\end{frame}

%% Slide with citation
\begin{frame}[t]
    \frametitle{Gathering Raw Data}
    Raw data was gathered from the following sources:
    \begin{itemize}
        \item Please cite this template as: \citet{GaudeckerEconProjectTemplates}
        \item Example data is taken from \url{https://www.stem.org.uk/resources/elibrary/resource/28452/large-datasets-stats4schools}
    \end{itemize}
    \note{~}
\end{frame}

% \begin{frame}[t]
%     \begin{figure}[H]

%         \centering
%         \includegraphics[width=0.5\textwidth]{../bld/python/figures/smoking_by_marital_status}

%         \caption{\emph{Python:} Model predictions of the smoking probability over the
%             lifetime. Each colored line represents a case where marital status is fixed to
%             one of the values present in the data set.}
%         \label{fig:python-predictions}

%     \end{figure}
% \end{frame}

%% Slide with a figure and text in two seperate columns
\begin{frame}{Point Patterns of Burglary Incidents}
    \begin{columns}
        % Column 1
        \begin{column}{0.5\textwidth}
            \begin{itemize}
                \item \textbf{Kernel density estimates:} Counting the number of points in a continuous way.
                \item Burglaries are centered around the city center.
            \end{itemize}
        \end{column}
        % Column 2    
        \begin{column}{0.5\textwidth}
            \begin{figure}
            \centering
                \includegraphics[width=1.0\textwidth]{burglary_hotspots.png}
                \caption{Burglary hotspots within greater London Area.}
            \end{figure}
        \end{column}
    \end{columns}
\end{frame}

%% Slide with a figure and text in two seperate columns
\begin{frame}{Spatial Clustering of Burglary Incidents}
    \begin{columns}
        % Column 1
        \begin{column}{0.5\textwidth}
            \begin{itemize}
                \item \textbf{DBSCAN:} density-based clustering
                \item \textbf{Assumption}: clusters are dense regions in space separated by regions of lower density
                \item Burglaries are centered around the city center.
            \end{itemize}
        \end{column}
        % Column 2    
        \begin{column}{0.5\textwidth}
            \begin{figure}
            \centering
                \includegraphics[width=1.0\textwidth]{burglary_clusters.png}
                \caption{Burglary clusters within greater London Area.}
            \end{figure}
        \end{column}
    \end{columns}
\end{frame}

%% Slide with a figure and text in two seperate columns
\begin{frame}{Spatial Weights Matrix}
    \begin{columns}
        % Column 1
        \begin{column}{0.5\textwidth}
            \begin{itemize}
                \item A spatial weights matrix is the way geographical space is formally encoded into 
                a numerical form so it is easy for a computer (or a statistical method) to understand.
                \item We can define a spatial weights matrix, such as contiguity, distance-based, 
                or block. We have used Queen contiguity matrix. 
                \item \textbf{Queen Contiguity Matrix:} For a pair of local authorities in the dataset to be 
                considered neighbours under this W, they will need to be sharing border or, in other words, “touching” each other to some degree. 

            \end{itemize}
        \end{column}
        % Column 2    
        \begin{column}{0.5\textwidth}
            \begin{figure}
            \centering
                \includegraphics[width=1.0\textwidth]{weights_matrix_ward.png}
                \caption{Burglary clusters within greater London Area.}
            \end{figure}
        \end{column}
    \end{columns}
\end{frame}

% \textbf{Spatial lag} is a spatially weighted average of the values of a 
%         variable in the neighbourhood of each observation.Once we have weights matrix,
%          we can look at the spatial lag of the burglary 
%         in the Greater London area

%% Slide with two figures and text 
\begin{frame}{Spatial Lag of Burglary Incidents}
    \textbf{Spatial lag} is a spatially weighted average of the values of a 
    variable in the neighbourhood of each observation.Once we have weights matrix,
    we can look at the spatial lag of the burglary in the Greater London area
    
    %You can see in Figure \ref{fig:images} that I have inserted two images, Figures \ref{fig:nature1} and \ref{fig:nature2}, that I can reference independently. More text, More text, More text, More text, More text,More text ,More text 
    \begin{figure}
        \centering
            \begin{subfigure}[t]{0.4\textwidth}
                \includegraphics[width=\textwidth]{burglary_ward.png}
                \caption{No. of Burglary Incidents (2019)}\label{fig:burglary}
            \end{subfigure} 
            \begin{subfigure}[t]{0.4\textwidth}
                \includegraphics[width=\textwidth]{burglary_ward_lag.png}
                \caption{Burglary Lag (2019)}\label{fig:burglary_lag}
            \end{subfigure}
        \caption{Burglary incidents and the corresponding lag on a ward level.}\label{fig:lag}
    \end{figure}
\end{frame}

%% Slide with two figures and text 
\begin{frame}{Moran's I}
    
    \begin{itemize}
        \item \textbf{Moran's I:} It is a measure of \textbf{spatial autocorrelation}. 
        \item \textbf{Assumption:} The data is normally distributed.
        \item \textbf{Null hypothesis:} The data is randomly distributed.
        \item \textbf{Alternative hypothesis:} The data is not randomly distributed.
    \end{itemize}
    %You can see in Figure \ref{fig:images} that I have inserted two images, Figures \ref{fig:nature1} and \ref{fig:nature2}, that I can reference independently. More text, More text, More text, More text, More text,More text ,More text 
    \begin{figure}
        \centering
            \begin{subfigure}[t]{0.4\textwidth}
                \includegraphics[width=\textwidth]{moran_scatter.png}
                \caption{Moran's Plot}\label{fig:moran_scatter}
            \end{subfigure} 
            \begin{subfigure}[t]{0.4\textwidth}
                \includegraphics[width=\textwidth]{moran_distribution.png}
                \caption{Distribution after random Sampling}\label{fig:moran_distribution}
            \end{subfigure}
        \caption{Moran's I and Global Spatial Autocorrelation}\label{fig:moran}
    \end{figure}
\end{frame}

%% Slide with two figures and text 
\begin{frame}{Resource Deprivation in Greater London Area}
    Resource deprivation scores on a LSOA level as shown in Figure \ref{fig:imd_scores_lsoa} 
    have been aggregated to a ward level for subsequent spatial regression analysis
    as shown in Figure \ref{fig:imd_scores_ward}.
    \begin{figure}
        \centering
            \begin{subfigure}[t]{0.4\textwidth}
                \includegraphics[width=\textwidth]{imd_scores_lsoa.png}
                \caption{Indices of Multiple Deprivation LSOA level}\label{fig:imd_scores_lsoa}
            \end{subfigure}
            \begin{subfigure}[t]{0.4\textwidth}
                \includegraphics[width=\textwidth]{imd_scores_ward.png}
                \caption{Indices of Multiple Deprivation on ward level}\label{fig:imd_scores_ward}
            \end{subfigure}
        \caption{English Indices of Multiple Deprivation (IMD)}\label{fig:imd}
    \end{figure}
\end{frame}


%% Slide with two figures and text 
\begin{frame}{Spatial Regression: Burglaries and Resource Deprivation}
    Number of burglaries a shown in in Figure \ref{fig:burglary_ward} and resource deprivation as shown 
    in Figure \ref{fig:imd_scores_ward} are used for spatial regression analysis on a ward level.
    \begin{figure}
        \centering
            \begin{subfigure}[t]{0.4\textwidth}
                \includegraphics[width=\textwidth]{burglary_ward.png}
                \caption{Indices of Multiple Deprivation LSOA level}\label{fig:burglary_ward}
            \end{subfigure}
            \begin{subfigure}[t]{0.4\textwidth}
                \includegraphics[width=\textwidth]{imd_scores_ward.png}
                \caption{Indices of Multiple Deprivation on ward level}\label{fig:imd_scores_ward}
            \end{subfigure}
        \caption{English Indices of Multiple Deprivation (IMD)}\label{fig:burg_imd}
    \end{figure}
\end{frame}

                
% Slide with table
\begin{frame}{Spatial Regression: OLS}
    \small\begin{table}[!h]
        \input{../bld/python/tables/model_spatial_ols_summary.tex}
        \caption{\label{tab:ols_summary} Estimation results of OLS regression.}
    \end{table}
\end{frame}

% Slide with table
\begin{frame}{Spatial Regression: ML\_Lag}
    \small\begin{table}[!h]
        \input{../bld/python/tables/model_spatial_ml_lag_summary.tex}
        \caption{\label{tab:ml_lag_summary} Estimation results of ML\_Lag regression.}
    \end{table}
\end{frame}

% Slide with table
\begin{frame}{Spatial Regression: ML\_Error}
    \small\begin{table}[!h]
        \input{../bld/python/tables/model_spatial_ml_error_summary.tex}
        \caption{\label{tab:ml_error_summary} Estimation results of ML\_Error regression.}
    \end{table}
\end{frame}


% % Slide with table
% \begin{frame}{Tables in Beamer}
%     \small\begin{table}[!h]
%         \input{../bld/python/tables/model_spatial_ml_error_summary.tex}
%         \caption{\label{tab:python-summary}\emph{Python:} Estimation results of the
%             linear Logistic regression.}
%     \end{table}
% \end{frame}

% %% Slide with a figure and text in two seperate columns
% \begin{frame}{Figure with other content}
%     \begin{columns}
%         % Column 1
%         \begin{column}{0.5\textwidth}
%                 Here I can explain in detail what the figure represents.
%         \end{column}
%         % Column 2    
%         \begin{column}{0.5\textwidth}
%             \begin{figure}
%             \centering
%                 \includegraphics[width=1.0\textwidth]{burglary_ward.png}
%                 \caption{A figure that is next to a certain explanation.}
%             \end{figure}
%         \end{column}
%     \end{columns}
% \end{frame}

% %% Slide with two figures and text 
% \begin{frame}{Subfigures in Beamer}
%     You can see in Figure \ref{fig:images} that I have inserted two images, Figures \ref{fig:nature1} and \ref{fig:nature2}, that I can reference independently. More text, More text, More text, More text, More text,More text ,More text 
%     \begin{figure}
%         \centering
%             \begin{subfigure}[t]{0.4\textwidth}
%                 \includegraphics[width=\textwidth]{burglary_ward.png}
%                 \caption{Image of the nature V1.}\label{fig:nature1}
%             \end{subfigure}
%             \begin{subfigure}[t]{0.4\textwidth}
%                 \includegraphics[width=\textwidth]{imd_scores_ward.png}
%                 \caption{Image of the nature V2.}\label{fig:nature2}
%             \end{subfigure}
%         \caption{Two images I want to insert.}\label{fig:images}
%     \end{figure}
% \end{frame}
                
% % Slide with table
% \begin{frame}{Tables in Beamer}
%     \small\begin{table}[!h]
%         \input{../bld/python/tables/model_spatial_ml_error_summary.tex}
%         \caption{\label{tab:python-summary}\emph{Python:} Estimation results of the
%             linear Logistic regression.}
%     \end{table}
% \end{frame}


\begin{frame}[allowframebreaks]
    \frametitle{References}
    \renewcommand{\bibfont}{\normalfont\footnotesize}
    \printbibliography
\end{frame}

\end{document}
