\documentclass[11pt, aspectratio=169]{beamer}
% \documentclass[11pt,handout]{beamer}
\usepackage[T1]{fontenc}
\usepackage[utf8]{inputenc}
\usepackage{textcomp}
\usepackage{float, afterpage, rotating, graphicx}
\usepackage{epstopdf}
\usepackage{longtable, booktabs, tabularx}
\usepackage{fancyvrb, moreverb, relsize}
\usepackage{eurosym, calc}
\usepackage{amsmath, amssymb, amsfonts, amsthm, bm}
\usepackage[
    natbib=true,
    bibencoding=inputenc,
    bibstyle=authoryear-ibid,
    citestyle=authoryear-comp,
    maxcitenames=3,
    maxbibnames=10,
    useprefix=false,
    sortcites=true,
    backend=biber
]{biblatex}

% for figures
\usepackage{subcaption}
\usepackage[english]{babel}

\usepackage{graphicx}
\graphicspath{{../bld/python/figures/}}

% for csv tables
\usepackage{siunitx,array,booktabs}
\usepackage{csvsimple}
\usepackage{adjustbox}

\AtBeginDocument{\toggletrue{blx@useprefix}}
\AtBeginBibliography{\togglefalse{blx@useprefix}}
\setlength{\bibitemsep}{1.5ex}
\addbibresource{refs.bib}

\hypersetup{colorlinks=true, linkcolor=black, anchorcolor=black, citecolor=black, filecolor=black, menucolor=black, runcolor=black, urlcolor=black}

\setbeamertemplate{footline}[frame number]
\setbeamertemplate{navigation symbols}{}
\setbeamertemplate{frametitle}{\centering\vspace{1ex}\insertframetitle\par}


\begin{document}

\title{Crime Data Mapping and Spatial Regression}

\author[Mudabbira Mushtary]
{
{\bf Mudabbira Mushtary}\\
{\small University of Bonn}\\[1ex]
}


\begin{frame}
    \titlepage
    \note{~}
\end{frame}

%% Slide with citation
\begin{frame}[t]
    \frametitle{Gathering Raw Data}
    Raw data was gathered from the following sources:
    \begin{itemize}
        \item Please cite this template as: \citet{GaudeckerEconProjectTemplates}
        \item Example data is taken from \url{https://www.stem.org.uk/resources/elibrary/resource/28452/large-datasets-stats4schools}
    \end{itemize}
    \note{~}
\end{frame}

% \begin{frame}[t]
%     \begin{figure}[H]

%         \centering
%         \includegraphics[width=0.5\textwidth]{../bld/python/figures/smoking_by_marital_status}

%         \caption{\emph{Python:} Model predictions of the smoking probability over the
%             lifetime. Each colored line represents a case where marital status is fixed to
%             one of the values present in the data set.}
%         \label{fig:python-predictions}

%     \end{figure}
% \end{frame}

%% Slide with a figure and text in two seperate columns
\begin{frame}{Figure with other content}
    \begin{columns}
        % Column 1
        \begin{column}{0.5\textwidth}
                Here I can explain in detail what the figure represents.
        \end{column}
        % Column 2    
        \begin{column}{0.5\textwidth}
            \begin{figure}
            \centering
                \includegraphics[width=1.0\textwidth]{burlary_ward.png}
                \caption{A figure that is next to a certain explanation.}
            \end{figure}
        \end{column}
    \end{columns}
\end{frame}

%% Slide with two figures and text 
\begin{frame}{Subfigures in Beamer}
    You can see in Figure \ref{fig:images} that I have inserted two images, Figures \ref{fig:nature1} and \ref{fig:nature2}, that I can reference independently. More text, More text, More text, More text, More text,More text ,More text 
    \begin{figure}
        \centering
            \begin{subfigure}[t]{0.4\textwidth}
                \includegraphics[width=\textwidth]{burlary_ward.png}
                \caption{Image of the nature V1.}\label{fig:nature1}
            \end{subfigure}
            \begin{subfigure}[t]{0.4\textwidth}
                \includegraphics[width=\textwidth]{imd_scores_ward.png}
                \caption{Image of the nature V2.}\label{fig:nature2}
            \end{subfigure}
        \caption{Two images I want to insert.}\label{fig:images}
    \end{figure}
\end{frame}
                
% Slide with table
\begin{frame}{Tables in Beamer}
    \small\begin{table}[!h]
        \input{../bld/python/tables/model_spatial_ml_error_summary.tex}
        \caption{\label{tab:python-summary}\emph{Python:} Estimation results of the
            linear Logistic regression.}
    \end{table}
\end{frame}



\begin{frame}[allowframebreaks]
    \frametitle{References}
    \renewcommand{\bibfont}{\normalfont\footnotesize}
    \printbibliography
\end{frame}

\end{document}
